Tissue classification plays a crucial role in the investigation of normal neural development, brain-behavior relationships, and the disease mechanisms of many psychiatric and neurological illnesses. Ensuring the accuracy of tissue classification is important for quality research and, in particular, the translation of imaging biomarkers to clinical practice. Assessment with the human eye is vital to correct various errors inherent to all currently available segmentation algorithms. Manual quality assurance becomes methodologically difficult at a large scale - a problem of increasing importance as the number of data sets is on the rise. To make this process more efficient, we have developed Mindcontrol, an open-source web application for the collaborative quality control of neuroimaging processing outputs. The Mindcontrol platform consists of a dashboard to organize data, descriptive visualizations to explore the data, an imaging viewer, and an in-browser annotation and editing toolbox for data curation and quality control. Mindcontrol is flexible and can be configured for the outputs of any software package in any data organization structure. Example configurations for three large, open-source datasets are presented: the 1000 Functional Connectomes Project (FCP), the Consortium for Reliability and Reproducibility (CoRR), and the Autism Brain Imaging Data Exchange (ABIDE) Collection. These demo applications link descriptive quality control metrics, regional brain volumes, and thickness scalars to a 3D imaging viewer and editing module, resulting in an easy-to-implement quality control protocol that can be scaled for any size and complexity of study.