\section{Software Design and Implementation}

\subsection{Design Principles}
Mindcontrol was developed with several design requirements. Mindcontrol must be easily accessible from any device, such as a Mac, Windows or even a tablet. Therefore, the best option was to develop a web application. Most tablets have limited storage capacity, so space-minimizing specifications were established. A dependence on cloud-based data storage was specified to accommodate large neuroimaging datasets without needing local storage. To efficiently store annotations and edited voxels, Mindcontrol only stores the changes to files, rather than whole-file information, on its database. Researchers must be able to QC outputs from any type of neuroimaging software package, so Mindcontrol was specified to flexibly accommodate any file organization structure, with configurable "modules" that can contain any type of descriptive statistics and 3D images. Mindcontrol configuration and database updates must require minimal Javascript knowledge, since Matlab/Octave, Python, R, and C are primarily used in the neuroimaging community for data analysis. Finally, changes to the database(like the addition of new images), changes in descriptive measures, and new edits/annotations, should be reflected in the application in real-time to foster collaboration.  


\subsection{Server Back-End Framework}

Mindcontrol is built with Meteor (\href{http://www.meteor.com}{http://www.meteor.com}), a full-stack javascript web-development platform. Meteor features a build tool, a package manager, the convenience of a single language (javascript) to develop both the front- and back-end of the application, and an abstracted implementation of full-stack reactivity. Data is transferred "over the wire" and rendered by the client (as opposed to the server sending HTML), which means that changes to the database automatically trigger changes to the application view. For example, as soon as a user finishes implementing QC procedures on an image and clicks "save", all other users can see the changes. A diagram of this process is provided in Figure \ref{fig:appstructure}.



