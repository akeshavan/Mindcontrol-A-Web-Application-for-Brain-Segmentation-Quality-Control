\section{Background}


Imaging biomarkers derived from MRI play a crucial role in the fields of neuroscience, neurology, and psychiatry. Estimates of regional brain volumes and shape features can track the disease progression of  neurological and psychiatric diseases such as Alzheimer's disease \cite{18632739,Vemuri_2010}, Parkinson's disease \cite{Silvia_Mangia_2013}, schizophrenia \cite{shenton2001review}, depression \cite{meisenzahl2011structural}, autism \cite{brambilla2003brain}, and multiple sclerosis \cite{Filippi_1995}. Given recent increases in data collection to accommodate modern precision-medicine approaches, assuring the quality of these biomarkers is vital as we scale their production. 

Various semi-automated programs have been developed to estimate MRI biomarkers. While these applications are efficient, errors in regional segmentation are inevitable, given several methodological challenges inherent to both technological and clinical implementation limitations. First, the quality of the MRI scan itself due to motion artifacts or scanner instabilities could blur and distort anatomical boundaries \cite{Blumenthal_2002,Pardoe_2016,Reuter_2015,Savalia_2016}. Differences in MRI hardware, software, and acquisition sequences also contribute to contrast differences and gradient distortions that affect tissue classification, which makes combining datasets across sites challenging \cite{keshavan2016power}. An additional source of error comes from parameter selection for segmentation algorithms; different parameter choices can translate to widely varying results \cite{Han_2006}. Furthermore, many MR segmentation algorithms were developed and tested on healthy adult brains; applying these algorithms to brain images of children, the elderly, or those with pathology may violate certain assumptions of the algorithm, resulting in drastically different results. 

Several quality assurance strategies exist to address segmentation errors. In one approach, researchers flag low-quality scans prior to analysis by viewing the data before input to tissue classification algorithms.  However, identifying "bad" datasets using the raw data is not always straightforward, and can be prohibitively time consuming for large datasets. Pre-processing protocols have been developed to extract metrics that can be viewed as a cohort-level summary from which outliers are selected for manual quality-assurance. For example, by running the Preprocessed-Connectomes Project’s Quality Assurance Protocol (PCP-QAP) \cite{shehzadpreprocessed}, researchers can view summary statistics that describe the quality of the raw data going into the algorithm and automatically remove subpar images. However, these metrics are limited because segmentation may still fail even if the quality of the scan is good. Another quality assurance strategy is to plot distributions of the segmentation output metrics themselves and remove any outlier volumes. However, without manual inspection, normal brains that naturally have very small or large estimates of brain size or pathological brains with valid segmentations may be inappropriately removed. Ideally, a link would exist between scalar summary statistics and 3D/4D volumes. Such a link would enable researchers to prioritize  images for labor-intensive quality control (QC) procedures; to collaborate and organize QC procedures; and to understand how scalar quality metrics, such as signal to noise ratio, relate to the actual image and segmentation. In this report, we present a collaborative and efficient MRI QC platform that links group-level descriptive statistics to individual volume views of MRI images.  

We propose an open source web-based brain quality control application called Mindcontrol: a dashboard to organize, QC, annotate, edit, and collaborate on neuroimaging processing. Mindcontrol provides an intuitive interface for examining distributions of descriptive measures from neuroimaging pipelines (e.g., surface area of right insula), and viewing the results of segmentation analyses using the Papaya.js volume viewer (\href{https://github.com/rii-mango/Papaya}{https://github.com/rii-mango/Papaya}). Users are able to annotate points and curves on the volume, edit voxels, and assign tasks to other users (e.g., to manually correct the segmentation of a particular image). The platform is pipeline agnostic, meaning that it can be configured to quality control any set of 3D volumes regardless of what neuroimaging software package produced it. In the following sections, we describe the implementation details of Mindcontrol, as well as its configuration for three open-source datasets, each with a different type of neuroimaging pipeline output. 

%Neuroimaging has also helped to understand cognitive development, from childhood, through adolescence\cite{giedd2010structural}, adulthood and senescence. 