\section{Background}


The imaging biomarkers derived from MRI play a crucial role in the fields of neuroscience, neurology and psychiatry. Estimates of regional brain volumes and shape features help track the disease progression of several neurological and psychiatric diseases, including Alzheimer's disease \cite{18632739,Vemuri_2010}, Parkinson's disease \cite{Silvia_Mangia_2013}, schizophrenia \cite{shenton2001review}, depression \cite{meisenzahl2011structural}, autism \cite{brambilla2003brain}, and multiple sclerosis \cite{Filippi_1995}, to name a few. Assuring the quality of these biomarkers is crucial as we scale their production, given recent increases in data collection to accommodate modern precision-medicine approaches. 

Multiple semi-automated programs have been developed to estimate such MRI biomarkers. While these programs are efficient, errors in regional segmentation frequently occur, stemming from multiple sources. First, the quality of the MRI scan itself due to motion artifacts or scanner instabilities could blur and distort anatomical boundaries \cite{Blumenthal_2002,Pardoe_2016,Reuter_2015,Savalia_2016}. Differences in MRI hardware, software, and acquisition sequences also contribute to contrast differences and gradient distortions that affect tissue classification, which makes combining datasets across sites challenging \cite{keshavan2016power}. An additional source of error comes from parameter selection for segmentation algorithms, where varying parameter choices give different results \cite{Han_2006}. Furthermore, if MR segmentation algorithms were developed and tested on healthy adult brains, using the algorithm on brain images of children, the elderly, or those with pathology, may violate certain assumptions of the algorithm, resulting in drastically different results. 

Several quality assurance strategies exist to address these segmentation errors. Researchers could flag bad quality scans before analysis by viewing the data before input to tissue classification algorithms, however, often this is not very straightforward, and gets too time consuming for large datasets. By running the Preprocessed-Connectomes Project’s Quality Assurance Protocol (PCP-QAP)\cite{shehzadpreprocessed}, researchers can view summary statistics that describe the quality of the raw data going into the algorithm and automatically flagged subpar images. However, these metrics are limited because segmentation may still fail even if the scan is good quality. Another quality assurance strategy is to plot distributions of the segmentation output metrics themselves and remove any outliers. However, without visual inspection, normal brains that naturally have very small or large estimates of brain size or pathological brains with valid segmentations may be inappropriately removed. Ideally, a link would exist between scalar summary statistics and 3D/4D volumes. Such a link would allow researchers to prioritize which images to implement quality control (QC) procedures, to collaborate and organize QC procedures, and to understand how scalar quality metrics, such as signal to noise ratio, relate to the actual image and segmentation. In this report, we propose a collaborative and efficient MRI QC solution that links group-level descriptive statistics with individual volume views of MRI images.  

We propose an open source web-based brain quality control application called Mindcontrol, which is a dashboard to organize, QC, annotate, edit, and collaborate on neuroimaging processing results. Mindcontrol provides an easy-to-use interface for examining distributions of descriptive measures from neuroimaging pipelines (e.g., surface area of right insula), and viewing the results of segmentation analyses using the Papaya.js volume viewer (https://github.com/rii-mango/Papaya). Users are able to annotate points and curves on the volume, edit voxels, and assign tasks to other users (e.g., to correct the segmentation of a particular image). The platform is pipeline agnostic, meaning that it can be configured to QC any set of 3D volumes regardless of what neuroimaging software package produced it. In the following sections, we describe the implementation details of Mindcontrol, as well as its configuration to three different open-source datasets, with three different types of neuroimaging pipeline outputs. 

%Neuroimaging has also helped to understand cognitive development, from childhood, through adolescence\cite{giedd2010structural}, adulthood and senescence. 