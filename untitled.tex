\section{Background}


The imaging biomarkers derived from MRI play a crucial role in the fields of neuroscience, neurology and psychiatry. Estimates of regional brain volumes and shape features help track the disease progression of several neurological and psychiatric diseases, including Alzheimer's disease \cite{Vemuri_2010}, Parkinson's disease \cite{Silvia_Mangia_2013}, schizophrenia\cite{shenton2001review}, depression\cite{meisenzahl2011structural}, autism\cite{brambilla2003brain}, and multiple sclerosis\cite{Filippi_1995}, to name a few. Assuring the quality of these biomarkers is crucial, especially with recent increases in data collection to accommodate modern precision medicine approaches. 

Errors in regional segmentation stem from multiple sources. First, the quality of the MRI scan itself due to motion artifacts or scanner instabilities could blur and distort anatomical boundaries. Differences in MRI hardware, software, and acquisition sequences also contribute to contrast differences and gradient distortions that make combining datasets across sites challenging \cite{keshavan2016power}. MR segmentation algorithms generally were developed on images with standardized sequence parameters, and deviations from those sequences may give different results. Also, MR segmentation algorithms generally were developed on healthy adult brains, and using the algorithm on non-adult brains may violate certain assumptions of the algorithm, resulting in drastically different results. Finally, random idiosynchratic behavior of the algorithm resulting from a combination of these factors could happen, too.

Several quality assurance strategies exist to address these segmentation errors. All of the errors mentioned above could be flagged before analysis by first ensuring data quality before the pipeline by simply viewing the segmentations, with the included viewers of the software packages that the segmentations come from, like freeview and fslview. Because this gets too time consuming for large datasets, entire pipelines have been developed, like the QAP protocol, to produce summary statistics to describe the quality of the raw data going into the algorithm. Another quality assurance strategy is to plot distributions of the output metrics themselves and remove any outliers. This gets dangerous for pathological brains those outliers may be valid segmentations, and you miss out on an important data point, which is why looking at actual images is so important. Ideally, a link would exist between scalar sumamry statistics and 3D/4D images. Some software packages provide summary reports with 2D images and summary statistics, like FSL, but the 2D images sometimes are not enough and are usually on an individual basis. 

%Neuroimaging has also helped to understand cognitive development, from childhood, through adolescence\cite{giedd2010structural}, adulthood and senescence. 