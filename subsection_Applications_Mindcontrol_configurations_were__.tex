\subsection{Applications}

Mindcontrol configurations were developed for selected data from the 1000 Functional Connectomes project (FCP), the consortium for reliability and reproducibility (CoRR), and the Autism Brain Imaging Data Exchange (ABIDE) Collection I. The FCP consists of 1414 resting state fMRI and corresponding structural datasets collected from 35 sites around the world  \cite{biswal2010toward}, which has been openly shared to the public. The purpose of the FCP collaboration is to comprehensively map the functional connectome, understand genetic influences on brain's structure and function, and understand how brain structure and function relate to human behavior \cite{biswal2010toward}. Segmentation of 200 selected anatomical images from FCP from Baltimore, Bangor, Berlin, ICBM, and Milwaukee was run with Freesurfer (recon-all) version 5.3.0 \cite{fischl2002whole} using the RedHat 7 operating system on IEEE 754 compliant hardware. Regional volumes of cortical, subcortical, and cerebellar regions were computed and averaged across hemispheres. Cortical areas and thicknesses were also computed and averaged across hemispheres. Scan dates were simulated in order to demonstrate the date histogram shown in figure \ref{fig:appstructure}B. The original, anonymized, T1-weighted images, along with the aparc+aseg output from Freesurfer, were uploaded to Dropbox for the purpose of visualization within Mindcontrol. The Mindcontrol database was populated with URLs to these images, along with their corresponding freesurfer segmentation metrics.

The purpose of the CoRR is to provide an open-science dataset to assess the reliability of functional and structural connectomics, by defining test-retest reliability of commonly used MR metrics, understanding the variability of these metrics across sites, and to establish a standard benchmark dataset to evaluate new imaging metrics \cite{Zuo_2014}. The Preprocessed-Connectomes Project's Quality Assurance Protocol (PCP-QAP) software was developed to provide anatomical and functional data quality measures, in order to detect low quality images before data processing and analysis \cite{shehzadpreprocessed}. PCP-QAP normative data for the CoRR study was downloaded from \href{http://raw.githubusercontent.com/preprocessed-connectomes-project/quality-assessment-protocol/master/poster_data/corr_anat.csv}{https://github.com/preprocessed-connectomes-project/quality-assessment-protocol}. The Mindcontrol database was populated with pointers to 2,963 CoRR structural images residing on an Amazon S3 bucket along with their corresponding PCP-QAP metrics. Some metrics include contrast-to-noise ratio, signal-to-noise ratio, voxel smoothness, percentage of artifact voxels, foreground-to-background energy ratio, and entropy focus criterion \cite{shehzadpreprocessed}. 

The overarching goal of the ABIDE initiative is to expedite the discovery of the neural basis of autism by providing open access to large, heterogeneous collection of structural and functional neuroimaging data, collected from over 23 institutions \cite{Di_Martino_2013}. The preprocessed connectomes project provides cortical thickness measures from the ANTs software package \cite{avants2009advanced} of the ABIDE dataset, along with summary statistics across regions of interests (ROIs) defined Desikan-Killiany-Tourville (DKT) protocol \cite{Klein_2012}. The Mindcontrol database was populated with pointer URLs to S3-hosted cortical thickness images and their corresponding ROI summary measures. The FCP, CoRR and ABIDE Mindcontrol configurations were hosted on Heroku (\href{http://www.heroku.com/}{https://www.heroku.com/})

%In response, the Autism Brain Imaging Data Exchange (ABIDE) initiative has aggregated functional and structural brain imaging data collected from laboratories around the world to accelerate our understanding of the neural bases of autism. With the ultimate goal of facilitating discovery science and comparisons across samples, the ABIDE initiative now includes two large-scale collections: ABIDE I and ABIDE II. 


\begin{itemize}
\item abide + ANTS
\end{itemize}