\section{Discussion}

Mindcontrol is a configurable neuroinformatics dashboard that links information visualization, descriptive stats, with scientific data visualization, MRI images and their overlays (segmentation or otherwise). The three example configurations in this report demonstrate the link between MRI quality metrics and raw data, the link between Freesurfer regional volumes and segmentation quality, and the link between antsCortical Thickness summary statistics and segmentation/thickness estimates on the volume. The platform is configurable, open-source, and software/pipeline agnostic, enabling researchers to configure it to their particular analyses. The dashboard allows researchers to assign editing tasks to others, who can then perform edits on the app itself. 

There have been considerable efforts in this field to ensure data quality on a large scale. The human connectome project's extensive informatics pipeline, which includes a database service, quality control procedures, and a data visualization platform, have been key to the project's success in collecting a large, high quality dataset \cite{Marcus_2013}. The Allen Brain Atlas offers a comprehensive genetic, neuroanatomical and connectivity web-based data exploration portal, linking an MRI viewer with data tables \cite{Sunkin_2012}. The open-source LORIS web-based data management system integrates an imaging viewer with extensive quality control modules\cite{Das_2012}. Mindcontrol supplements these efforts by providing a lightweight and extensible data management and visualization system with the added ability to perform edits and curate annotations within the application. 

Mindcontrol is actively being used at UCSF to study imaging biomarkers of multiple sclerosis (MS) disease progression in a 12-year longitudinal cohort of over 500 patients. MRI plays a crucial role in the diagnosis of MS due to its sensitivity to the white matter lesions characteristic of this disease \cite{ge2006multiple}.  Both the location and number of lesions are used in the diagnostic criteria of MS \cite{mcdonald2001recommended}; their evolution over time is considered a proxy for disease progression, and is closely monitored as a key outcome measure in clinical trials \cite{ge2000glatiramer}. Researchers have identified numerous imaging biomarkers of disease progression, including cortical atrophy \cite{fisher2008gray} and sensitive lesion quantification methods from FLAIR sequences \cite{Schmidt_2012}. 

Currently, Mindcontrol is being used as a data management dashboard and a quality control system for pial surface editing of Freesurfer outputs, and plays a role in the semi-automated identification of multiple sclerosis lesions. The Lesion Segmentation Toolbox automates the detection of FLAIR hyper-intense lesions, which is useful because it prevents user-bias of the lesion boundary \cite{Schmidt_2012}. However, because our acquisition protocols differ from those of the LST developers, manual intervention by neuroradiologists is often necessary to correct false positives and false negatives in our dataset, which can then be used to tune LST parameters. Mindcontrol's point annotation feature is used by neuroradiologists to mark the center of a false positive lesion. The curve annotation feature is used to roughly and quickly note the outline of a false negative lesion. Custom python scripts use these annotations to produce a corrected lesion segmentation, which is then used to tune the parameters of the LST.

\begin{itemize}

\item Freesurfer has a pretty involved thing
    \begin{itemize}
    \item Useful for:
    \item Freesurfer editing
    \item fMRI pipeline intermediates
    \item Checking a huge amount of files w/ lots of people
    \end{itemize}
\item uses in our MS lab: dura editing, lesion segmentation toolbox + manual intervention

\end{itemize}



