The imaging view is shown in Figure \ref{fig:imagingview}. The left-side column includes a section to mark an image as ``Pass'', ``Fail'', ``Edited'', or ``Needs Edits'' and to provide notes. The status bar at the top-left portion updates instantaneously with information on which user checked the image, the quality status of the image, and when it was last checked. Users are also able to assign edits to be performed by other users on the system; for example, a research assistant can perform a general QC and assign difficult cases to a neuroradiologist. On the right-hand side, the Papaya.js viewer (\href{http://rii-mango.github.io/Papaya/}{http://rii-mango.github.io/Papaya/}) is used to display the NifTI volumes of the original data and FreeSurfer segmentations. Images must be hosted on a separate server or a content delivery network (CDN) and the Mindcontrol database populated with URLs to these images.
