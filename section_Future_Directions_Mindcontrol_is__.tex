\section{Future Directions}

Mindcontrol is being actively developed to incorporate new features that will improve outlier detection, efficiency, and collaboration. New information visualizations to detect outliers include: scatter plots to compare two metrics against each other, and a longitudinal view of a single-subject trajectory for a given metric to detect uncharacteristic temporal changes. New scientific data visualizations are planned using the BrainBrowser library \cite{Sherif_2015} to display cortical surfaces. A beta version of real-time collaborative annotations, where two users can annotate the same image and see the edits of the other user as they occur, is in the testing phase. 

Currently, configuring Mindcontrol involves creating one JSON file to describe the different modules and another JSON file to populate the Mongo database with pointers to images and their scalar metrics. In the future, this process could be streamlined by creating a Mindcontrol configuration for datasets with a standardized folder structure, like the Brain Imaging Data Structure (BIDS) \cite{Gorgolewski_2016}, and their BIDS-compliant derivatives \cite{gorgolewski2016bids}. Further development of Mindcontrol will include the flexible importing of additional scalar metrics, such as measures of structural complexity, calculated by third-party toolboxes developed to complement standard analysis pipelines \cite{madan2016,madan2017}. This will enable researchers to collaborate on the same dataset by uploading metrics from their newly developed algorithms, and will enable them to easily explore their results in the context of metrics contributed by others. Finally, Mindcontrol has the potential to be a large-scale crowd-sourcing platform for segmentation editing and quality control. We hope the functionality, ease-of-use, and modularity offered by Mindcontrol will help to improve the standards used by studies relying on brain segmentation. We are moving into a world of big data, with an unprecedented amount of information available on each individual patient. This explosion of information availability has the capacity to revolutionize research and treatment of complex diseases, such as MS. To harness the potential of these big data initiatives, infrastructure such as this multiparametric contextualization and visualization tool is crucial as we move forward into this new phase of interdisciplinary translational research.