\section{Future Directions}

Mindcontrol is being actively developed with new features. New information visualizations to detect outliers are being developed, in the form of scatter plots to compare two metrics against each other, and a longitudinal view for the trajectory of single subject for a given metric. New scientific data visualizations are planned, by using using the BrainBrowser library \cite{Sherif_2015} to display cortical surfaces. There is a beta version of real-time collaborative annotations, where two users can annotate the same image and see the edits of the other as they occur. 

Currently, configuring Mindcontrol involves creating a JSON file to describe the different modules and another JSON file to populate the Mongo database with pointers to images and their scalar metrics. In the future, this process could be streamlined by creating a Mindcontrol configuration for datasets with a standardized folder structure, like the Brain Imaging Data Structure (BIDS) \cite{Gorgolewski_2016} and their derivatives \cite{gorgolewski2016bids}. Further development of Mindcontrol will include the flexible importing of additional scalar metrics, such as measures of structural complexity, calculated by third-party toolboxes developed to complement standard analysis pipelines \cite{madan2016,madan2017}. This will enable researchers to collaborate on the same dataset by uploading metrics from their newly developed algorithms, and will allow them to explore their relationship with metrics contributed by others. Finally, Mindcontrol has the potential to be a large-scale crowd-sourcing platform for segmentation editing and quality checking. We hope the functionality, ease-of-use, and modularity offered by Mindcontrol will help to improve the standards used by studies relying on brain segmentation.